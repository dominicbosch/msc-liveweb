\chapter{Introduction}
% NOTES / TODOs:
%
% Take from preparation
%
% Definitions! need to be in text
% WebAPI
% Engine
% Event Poller
% Action Invoker
% Rules
% Rule Language (JSON Object?)
% Reactivity
% Programmablity of the web


% SYSTEM??? what do we call it in the theory part?
% what do we call it when we talk about the prototype system??
% Do I reference my own work?


The fast evolving web has brought up a trend towards easy to master interfaces to services, the so called WebAPIs.
They do not only provide access to mere services but whole applications that allow access over WebAPIs.
These trending WebAPIs benefit from a RESTful architecture which predominantly uses HTTP and thus relies on the most basic and powerful operations and the basis of the Web itself, the HTTP protocol. 

quick (handling/mastering) accessible services and even whole web applications through so called Web APIs.
WebAPIs provide powerful tools to govern data and functionality in the web independently of any user interface from the service provider.
The relatively
Allowing access to these services via API is increasingly popular and allows to mash up these services

Practically all services flood the user with events
The web should be event driven, that's why we need an engine that deals with events and makes the web reactive
There's still the challenge of filtering
What's important to whom
Plus the user needs to have tools to combine and add programmability to the combination,( such as conditions, selection of provided arguments and so on)



\section{Related Work}
% NOTES / TODOs:
%
% Zapier, IFTTT
% import.io
% Show kynetix?

% Grunt to build project
% jsprime to make code safe
% stylus for css
% (Categorized section at the end about existing rule languages)

\subsection{WebAPI Mashups}
Mashups combine information and functionality of more than one resource in a single place.
The mashing up of such resources allows new points of view on data, or even ways to interact with them.
Simple functions are combined into more powerful ones which influence data and services in a way their founders eventually didn't even think of.
They have been developped ever since services in the web started to exist and were accessible in a more or less convenient way.
One of the earliest inventors of such a webservice mashup is Paul Rademacher.
In the same year after Google Maps came up in 2005, he invented a site\cite{wwwRademacherOne,wwwRademacherTwo} that displayed Craigslist houses on a Google Map.
With no Google Maps API at that time, he needed time and skills to reverse engineer Google Map's functionalities.


A large number of such "static" mashups were and are still developped.
They are static in the way that they aggregate a fixed (and mostly low) number, of either data or functionality resources, to provide an enhanced resource in a specialized domain.
Of course Mashups can be mashed up again, to provide even more sophisticated functionality and data.
Some latest example Mashups, taken from the ProgrammableWeb\cite{wwwProgrammableWeb} directory, are:

\begin{itemize}
  \item Wifi and Plugs\cite{wwwWifiAndPlugs}: MapBox, Google Docs and Import.io API's used to display where Wi-Fi and plugs are available in London.
  \item MapLight\cite{wwwMapLight}: GovTrack.us and OpenSecrets API's used to combine political results with financial contributions to show how capital contributions affect voting.
  \item Shared Count\cite{wwwSharedCount}: Facebook, LinkedIn, Pinterest and Twitter API's used to display informations about how well spread a URL is on social media sites.
\end{itemize}

In the past few years, research and development for platforms to allow users to flexibly mashup WebAPIs got attention.
With IFTTT and Zapier, two platforms have evolved out of this process.
Users that register on those platforms are provided with a multitude of WebAPI functions that act as event triggers and such that are used to execute actions.
The user is then free to combine these event triggers and actions in the way it suits best, creating helpful WebAPI mashups on their own.

% Since this is a quite new field the why and how is hidden from the research.
% you need events
% you need rules / rule language