
\newglossaryentry{infospace}{
  name=Information Space,
  description={\textit{"[...] is a set of concepts and relations among them held by an \textrm{Information System}. \textrm{Information Space} is produced by a set of known procedures, and is changed through intentional manipulation of its content"}\cite{newby1996metric}}
}
\newglossaryentry{infosystem}{
  name=Information System,
  description={is a network of software and hardware components that support collection, filtering, storing, processing and distribution of data}
}
\newglossaryentry{mashup}{
  name=Mashup,
  description={A \textrm{Web Application} that weaves two or more different \textrm{Web APIs} together to provide a new perspective on data}
}

\newglossaryentry{semanticweb}{
  name=Semantic Web,
  description={Tim Berners-Lee's vision of the machine-readable Web through standard data formats and semantic metadata descriptions on top of the resources via \textrm{RDF} which turn the Web into a structured data collection}
}

\newglossaryentry{webapplication}{
  name=Web Application,
  description={An application which runs in the browser. Often a user interface to an application sitting on a server. Single Page Applications (\textrm{SPA}), a subset of the \textrm{Web Applications}, access \textrm{Web Resources} over asynchronous calls to the server while the user is interacting with the application}
}
\newglossaryentry{webapi}{
  name=Web API,
  description={An \textrm{Application Programming Interface} to either a \textrm{Web service} or the browser, used for application to application communication}
}
\newglossaryentry{webhook}{
  name=Webhook,
  description={A server-side \textrm{Web API} which accepts a \textrm{URI} that is used as a callback to push events to an external \textrm{Web Resource}}
}
\newglossaryentry{webService}{
  name=Web Service,
  description={A collection of \textrm{SOAP} related \textrm{Web service} (note the lower-case word \textrm{service}) standards which are widely adopted and developed in the industry. Also called \textrm{"WS-*" Web Services} or the \textrm{Big Web Services}}
}
\newglossaryentry{webservice}{
  name=Web service,
  description={An interface for communication between applications over a network. They can provide access to and control over \textrm{Web Resources}. \textrm{Web services} are also called \textrm{services on the Web} or just \textrm{services}}
}
\newglossaryentry{webofthings}{
  name=Web of Things,
  description={An evolution of the \textrm{Internet of Things}, which describes the integration of smart things (e.g. sensors, embedded devices or digitally enhanced objects) into the Internet. The \textrm{Web of Things} is the adoption of the \textrm{REST} architectural style to the smart things in order to enable uniform access to these loosely coupled entities}
}
\newglossaryentry{webresource}{
  name=Web Resource,
  description={Anything in the Web which can be identified, addressed and handled. Identification and addressing is often done over \textrm{URIs}. \textrm{Web Resources} and their semantic properties are described using \textrm{RDF} in the \textrm{Semantic Web}}
}
\newglossaryentry{web}{
  name=Web,
  description={A common term referring to the current state of the \textrm{World Wide Web}, which underlines the use of recent dynamic technologies to enhance Webpages into \textrm{Web Applications}, also called \textrm{Web 2.0}}
}
\newglossaryentry{www}{
  name=World Wide Web,
  description={Tim Berners-Lee's vision of interlinked hypertext documents which are accessed over the Internet via browser and allow the navigation through a global information universe. The term \textrm{World Wide Web} is often referred to as the first stage of the \textrm{Web}, the \textrm{Web 1.0}}
}

\newacronym{ced}{CED}{Complex Event Detection}
\newacronym{cep}{CEP}{Complex Event Processing}
\newacronym{corba}{CORBA}{Common Object Request Broker Architecture}
\newacronym{cms}{CMS}{Content Management System}
\newacronym{dom}{DOM}{Document Object Model}
\newacronym{eca}{ECA}{Event-Condition-Action}
\newacronym{eda}{EDA}{Event-Driven Architecture}
\newacronym{esb}{ESB}{Enterprise Service Bus}
\newacronym{html}{HTML}{Hypertext Markup Language}
\newacronym{http}{HTTP}{Hypertext Transfer Protocol}
\newacronym{iaas}{IaaS}{Infrastructure as a service}
\newacronym{idl}{IDL}{Interface Definition Language}
\newacronym{iiop}{IIOP}{Internet Inter-\textrm{ORB} Protocol}
\newacronym{json}{JSON}{JavaScript Object Notation}
\newacronym{kr}{KR}{Knowledge Representation}
\newacronym{kre}{KRE}{Kinetic Rules Engine}
\newacronym{krl}{KRL}{Kinetic Rule Language}
\newacronym{orb}{ORB}{Object Request Broker}
\newacronym{paas}{PaaS}{Platform as a service}
\newacronym{rest}{REST}{Representational State Transfer}
\newacronym{rewerse}{REWERSE}{Reasoning on the Web with Rules and Semantics}
\newacronym{rdf}{RDF}{Resource Description Framework}
\newacronym{rdftl}{RDFTL}{\textrm{RDF} Triggering Language}
\newacronym{rpc}{RPC}{Remote Procedure Call}
\newacronym{ruleml}{RuleML}{Rule Markup Language}
\newacronym{saas}{SaaS}{Software as a service}
\newacronym{soa}{SOA}{Service-Oriented Architecture}
\newacronym{soap}{SOAP}{Simple Object Access Protocol}
\newacronym{uri}{URI}{Uniform Resource Identifiers}
\newacronym{wsdl}{WSDL}{Web Service Description Language}
\newacronym{xml}{XML}{Extensible Markup Language}
\newacronym{xmlrpc}{XML-RPC}{\textrm{XML} - Remote Procedure Call}


