
\chapter{Introduction}

The web is an ever growing entity, in all aspects that it covers.
It surrounds a growing number of human beings in their daily life and starts to overwhelm them with an information flood.
One example of this information flood is the one hundred hours of video material that is uploaded to youtube\cite{wwwYoutube} every single minute.
An increasing number of bigger computing centers, but also ever smaller devices provide more data and functionality, both in quantity and complexity.
Also, a growing fraction of these devices have access to the web, which means they are potential data and functionality resources.
A white paper\cite{citeulike:12243016} estimates that 200 million devices were connected to the internet in the year 2000.
They also estimate that this number raised up to approximately 10 billion connected devices in 2013.
Further more, they expect this number to grow up to 50 billion connected devices by the year 2020.
This alone shows how strong growth of potentially data and functionalty delivering devices is growing currently and in the near future.
Other recent research\cite{conf/icws/HuangFT12}\cite{wwwProgrammableWebResearch} has shown that the number of accessible Web APIs follows power law distribution and thus provides an ever growing source of data and functionality.
The ever smaller devices that make up the "Internet of Things"\cite{Weber2010} today, are also capable of spreading the ubiquitous access to the web, e.g. through mobile devices.
 % and their push notifications, users get flooded with informations and, in turn, also produce increasingly more information.

The human being requires tools to get the right information in the right situation at the right place and also to automate tasks
user-centric reactions to allow a personalization of the information flood
Governing the web's information flood is getting more difficult and even fairly impossible for human beings with the tools given.


It becomes important to the individual to be able to filter out personally important bits and pieces.
Basic filters are often available but they do not allow smart filtering in any way.
Also, apart from smart filtering of information, people should have the possibility to aggregate important information in their desired place and in a way it's most useful for themselves.
Such an aggregation implies access to services that consume data and produce an output, be it a data answer or storage.
This means the user should get access to data and functionality services in a way that she can combine the possibilities in a suitable way to generated the most valuable output for her.
Even if the web service access gets simpler these days, the average user is not able to weild them.
The challenge to provide users with ways to handle these already simpler accessible services, called Web API's has received a notable amount of attention over the last few years.

% % TODO Reactivity!! allgemein
% \index{Reactivity}

% % TODO Programmability of the Web
% \index{Programmability}
%  -> in general, both should go into related work?

% TODO Information Space\cite{conf/trec/Newby96}
% Information space is a set of concepts and relations among them held by an information system.
% Information space is produced by a set of known procedures, and is changed through intentional
% manipulation of its content.

It is a promising research field that leads towards reactivity in the web through programmability.
Currently somebody that want's to program the web, requires deep knowledge of the required services and their functionality.
There are a few possibilities in the web that go towards easing the programmability of the web, but they are either complicated to weild themselves or mere data copy or mashup tasks.
Our research in this thesis is about easing the programmability of the web and therefore to achieve the reactive web. 
