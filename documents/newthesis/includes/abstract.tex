\chapter*{Abstract}
% The research focus (i.e. statement of the problem(s)/research issue(s) addressed);
% The research methods used (experimental research, case studies, questionnaires, etc.);
% The results/findings of the research; and
% The main conclusions and recommendations
The Web is a rapidly growing information universe, consisting of \textrm{\glspl{infosystem}} that provide access over heterogeneous services.
% TODO intro to Web?
The majority of those \textrm{\glspl{infosystem}} are dynamic and changes in their \textrm{\gls{infospace}} can be regarded as events, that can be detected.
But such changes can also be looked at as actions, when imposed onto the \textrm{\gls{infospace}}.
If appropriate services exist to access such an \textrm{\gls{infosystem}} for read and write operations, we are able to orchestrate it.
By adopting the \textrm{\acrlong{eca}} paradigm to \textrm{\gls{infospace}}, we are able to introduce an event-based conceptual model, which allows the detection of events and the dispatching of actions according to predefined rules, thus imposing reactivity on top of \textrm{\glspl{infosystem}}.
This model is not limited to the Web, but can include any in a way accessible \textrm{\glspl{infosystem}}.
In our work we introduce a prototype system which uses the Web's programmability to impose reactivity on top of it.
Through our prototype it is possible to orchestrate \textrm{\glspl{webresource}} based on an event-driven architecture, a method which pushes towards the vision of real-time reactive \textrm{\glspl{infosystem}}.
% TODO introduce WEbhooks
% TODO tell about examples
