\chapter*{Abstract}
The \textrm{\gls{web}} is a rapidly growing information universe, consisting of \textrm{\glspl{infosystem}} that provide access to heterogeneous services.
A large part of those \textrm{\glspl{infosystem}} are dynamic.
Changes in their \textrm{\gls{infospace}} trigger events, which can be detected.
Moreover, such changes can also be imposed onto the \textrm{\gls{infospace}} over the appropriate services.
If appropriate services exist to access such an \textrm{\gls{infosystem}} for read and write operations, we are able to orchestrate it.
By adopting the \textrm{\acrlong{eca} (\acrshort{eca})} paradigm to \textrm{\glspl{infosystem}}, we are able to introduce an event-based conceptual model.
This model allows the detection of events and the dispatching of actions according to predefined rules, thus imposing reactivity on top of or between \textrm{\glspl{infosystem}}.
Current approaches that use the \textrm{\acrshort{eca}} paradigm focus on action imposition on local storage, while we aim to impose actions on the hetereogeneous services of existing \textrm{\glspl{infosystem}}.
This model is not limited to the \textrm{\gls{web}}, but can include any accessible \textrm{\glspl{infosystem}}.

In our work we introduce a prototype system, which uses the \textrm{\gls{web}}'s programmability to impose reactivity on top of it.
We also underline the importance of the, currently little, support of \textrm{\glspl{webservice}} for event callback addresses, the so called \textrm{\glspl{webhook}}.
They are the only way for real-time event delivery to remote systems and free them from expensive polling for changes.
Through our prototype it is possible to orchestrate \textrm{\glspl{webservice}} based on an \textrm{\acrlong{eda} (\acrshort{eda})}, a method which pushes towards the vision of real-time reactive \textrm{\glspl{infosystem}}.
We list some example use cases for our conceptual model, as well as use cases that have been implemented in our prototype system.
