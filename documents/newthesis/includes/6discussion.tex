
\chapter{Conclusions \& Future Work}
% what reached, how to go on?
% Why is our system required, compare to servlets in terms of dynamic loading? user specific
% in the future what about somebody discovering our system as worthy to push events to? -> needs detection of new events
% We are taking a step further and allow not only the chaining up of several remote ECA engines, but also the invocation of actions on any arbitrary Web accessible service.

% In the future we could limit access to information systems to REST URI's and thus incorporate such calls into the language.
% but still we would need to be able to add logic, thus programming modules

%
% Temperature warning, using import.io
%http://khanlou.com/2014/03/model-view-whatever/
% Transactions in businesses, find use case. how would we lose events?
% Transactions for business. find use case.and explain how we would loose events.


% TODO Future Work
%
% CEP:

% Twelve Theses on Reactive Rules for the Web~\cite{10.1007-11896548_63}:
% This article investigates issues of relevance in designing high-level
% programming languages dedicated to reactivity on the Web. It presents
% twelve theses on features desirable for a language of reactive rules tuned
% to programming Web and Semantic Web applications.
% MY NOTES: they argue with soap and expect it to stay. They expect Web sites to inform each other about update requests. we are taking one step further and provide events to whomeever wants to receive them while being greedy about receiving events. give them to meeeeh my prciouzzz eventz!
% 1. High-level reactive languges are needed, ECA rules well-suited to specify reactivity
% 2. Reactive Web rules should be processed locally and act globally through event-based communication and access to persistent Web data
% 3. Events are best exchanged directly between Web sites in a push manner
% 4. Events are volatile data and should be kept distinct from persistent data.
% 5. Recognizing composite events is essential for a reactive Web language. Composite events are conveniently specified by (event) queries. There are (at least) four complementary dimensions to event queries: data extraction, event composition, temporal conditions, and event accumulation --> Future Work (CEP), data extraction already implemented
% 6. A data-driven, incremental evaluation of event queries is the approach of choice
% 7. Data from persistent Web resources plays an essential role for Web reactivity. A reactive language thus should embed or build upon a Web query language.
% 8. The Web is a dynamic, state-changing system. Reactions to state changes (events) through reactive rules are state-changing actions such as updates to persistent data. Reactive rules are needed where compound actions can be constructed from primitive actions.
% 9. Development and maintenance of reactive rule programs can be considerably supported by structuring mechanisms such as: branching in rules, deductive rules for event queries and Web queries, procedural abstractions for actions, and grouping of rules. --> Future work, attach several conditions and their actions branches to one event instead of creating rules for each of them. EC^nA^n. (Though procedural abstractions for actions already implemented, defining complex actions to be reused by rules)
% 10.Identity of data items is an issue for reactive languages due to their ability to react to changes of data objects on the Web. (we had surrogate identity though discarded the concept again... stupido)
% 11. Meta-programming and meta-circularity, that is, the ability to use rules to exchange and evaluate (other) rules, are needed in some important cases. (quite artificial for our scenarios)
% 12.Reactivity in the Web’s open and uncontrolled world requires language support for authentication, authorization, and accounting. (phew... let's let others go there)




% We have seen that the ECA approach is already a powerful one to make the Web reactive.
% CEP will result in an approach where events are not just processed when they are entering the system and evaluated against rules, but these events would need to be stored for quite a long time.
% Also the rules will not all be checked for each event but they are subject to a scheduler.
% It can be decided when and how often a rule is evaluated and all events will be checked at these point in times, whether they are candidates for firing the rule.
% A future improvement of this could be to adopt Complex Event Processing (CEP).
% This would mean that several events could be stored in a rule and be evaluated in terms of time constraints.
% Through this more complex events can be created as a result of several atomic events which would lead into semantically more complex events.


% TODO pathologische beispiele
% Endless loops -> child_process to be killed when not responding. what about async callbacks?

% Scheduler

% as long as we do not limit ourselves e.g for RESTful access to services we can't get it into a webquery language aight?



% RDFTL: The condition part is a query which de-
% termines if the information system is in a particular state, in which case the rule fires.

% TODO Condition evaluation on other resources? i.e. make a request to a remote site and evaluate it? -> also achievable through composite events
% Conditions evaluators could also be modules that are used to check certain states of information systems.
% Since we already have a lot of flexibility through our Event Triggers or Action Dispatchers this makes for us only sense for a future approach where CEP rules consist of accesses to REST interfaces, thus rule definitions get compiled into certain behaviour. such as web queries
% TODO Condition evaluation on other resources? i.e. make a request to a remote site and evaluate it? -> also achievable through composite events
% show rule engines, alsoo CEP, describe event composition through templates and why we don't do it

% automatic detection of new events within the system and information about them to the user
% the system needs to learn about events

% If we go to the semantic web we could incorporate RDF queries in order to allow smart event distinction

% web resurce (URI) can be data but also well defined service through REST. incorporating web resources into model 
% ince we're using URIs we could also set RDF logics on top of our event identification

% TODO forgot any important paper from the preparation report?